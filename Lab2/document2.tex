\documentclass[10pt]{beamer}
\usepackage[utf8]{inputenc}
\usepackage[russian]{babel}
\usetheme{Szeged}
\usecolortheme{beaver}

\title{Практика использования возможностей \LaTeX}
\subtitle{Презентация в \LaTeX  \ с помощью \textbf{Beamer}}
\author{Кураленко Антон Сергеевич}
\institute{Институт физико-математических наук и информационных технологий БФУ им. И. Канта}

\date{24 июня 2020 года}
\begin{document}

\begin{frame}
\titlepage
\end{frame}

\begin{frame}
	\frametitle{Описание курсовой работы}

	\begin{enumerate}
		\item {\bf Задачи курсовой}:
		\begin{enumerate}
			\item Освоить систему верстки \LaTeX.
			\item Использование распределенных систем управления версиями на примере GIT.
			\item Получить первичные профессиональные умения и навыки, в том числе первичные умения и навыки научно-исследовательской деятельности.
		\end{enumerate} 
		\item {\bf Методы решения}:
		\begin{enumerate}
			\item Создание репозитория на любом из бесплатных веб-сервисов систем конроля версий.
			\item в репозитории создать две папки: ''Lab 1'' , ''Lab 2''.
			\item Установление \LaTeX \ и (опционально) редактор для \LaTeX.
			\item Выполнение лабораторных работ и их редактирование.
		\end{enumerate} 
	\end{enumerate} 

\end{frame}

\begin{frame}
\frametitle{Результаты курсовой}
\begin{flushleft}
В результате учебной практики в 2020 году я познакомился с программой \LaTeX, ее основными инструментами и освоил систему ее верстки. Данное задание помогло мне глубже освоить английский язык, дало умение правильно редактировать научный текст, в частности, с математическим уклоном. Применил на практике совершенно новые для себя системы управления версиями на примере \textbf{GIT} .
\end{flushleft}
	
\end{frame}

\end{document}