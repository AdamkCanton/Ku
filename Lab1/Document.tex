 \documentclass[11pt]{article}
\usepackage{amsmath,amssymb,amsthm}
\usepackage{algorithm}
\usepackage[noend]{algpseudocode} 

%---enable russian----

\usepackage[utf8]{inputenc}
\usepackage[russian]{babel}

% PROBABILITY SYMBOLS
\newcommand*\PROB\Pr 
\DeclareMathOperator*{\EXPECT}{\mathbb{E}}


% Sets, Rngs, ets 
\newcommand{\N}{{{\mathbb N}}}
\newcommand{\Z}{{{\mathbb Z}}}
\newcommand{\R}{{{\mathbb R}}}
\newcommand{\Zp}{\ints_p} % Integers modulo p
\newcommand{\Zq}{\ints_q} % Integers modulo q
\newcommand{\Zn}{\ints_N} % Integers modulo N

% Landau 
\newcommand{\bigO}{\mathcal{O}}
\newcommand*{\OLandau}{\bigO}
\newcommand*{\WLandau}{\Omega}
\newcommand*{\xOLandau}{\widetilde{\OLandau}}
\newcommand*{\xWLandau}{\widetilde{\WLandau}}
\newcommand*{\TLandau}{\Theta}
\newcommand*{\xTLandau}{\widetilde{\TLandau}}
\newcommand{\smallo}{o} %technically, an omicron
\newcommand{\softO}{\widetilde{\bigO}}
\newcommand{\wLandau}{\omega}
\newcommand{\negl}{\mathrm{negl}} 

% Misc
\newcommand{\eps}{\varepsilon}
\newcommand{\inprod}[1]{\left\langle #1 \right\rangle}

 
\newcommand{\handout}[5]{
  \noindent
  \begin{center}
  \framebox{
    \vbox{
      \hbox to 5.78in { {\bf Научно-исследовательская практика} \hfill #2 }
      \vspace{4mm}
      \hbox to 5.78in { {\Large \hfill #5  \hfill} }
      \vspace{2mm}
      \hbox to 5.78in { {\em #3 \hfill #4} }
    }
  }
  \end{center}
  \vspace*{4mm}
}

\newcommand{\lecture}[4]{\handout{#1}{#2}{#3}{Scribe: #4}{Теория сравнений #1}}

\newtheorem{theorem}{Теорема}
\newtheorem{lemma}{Лемма}
\newtheorem{definition}{Определение}
\newtheorem{corollary}{Следствие}
\newtheorem{fact}{Факт}

% 1-inch margins
\topmargin 0pt
\advance \topmargin by -\headheight
\advance \topmargin by -\headsep
\textheight 8.9in
\oddsidemargin 0pt
\evensidemargin \oddsidemargin
\marginparwidth 0.5in
\textwidth 6.5in

\parindent=0.2in
\parskip 0ex

\begin{document}
	
\lecture{}{Лето 2020}{}{Кураленко Антон}
 где $ d_{i}= a_{i}-c_{i}$ для $i=0,1,\ldots,m$. Поскольку два представления для $ N $ преполагаются разными, мы должны иметь $ d_{t} \ne 0 $ для некоторого значения $ i $. Возьмем $ k $ как наименьший индекс, для которого $ d_{k} \ne 0 $. Тогда \[ 0=d_{m}b^{m}+\ldots+d_{k+1}b^{k+1}+d_{k}b^{k} \] и таким образом, после деления на $b^{k}$, \[ d_{k}=-b(d_{m}b^{m-k-1}+\ldots+d_{k+1}). \] Это говорит нам о том, что $ b \mid d_{k} $. Теперь неравенства $ 0\leq a_{k} < b  $ и $ 0\leq c_{k} < b  $ приводят к $ -b< a_{k}- c_{k} < b  $, или $ |d_{k}|<b  $. Единственный способ согласования условий $ b \mid d_{k} $ и $ |d_{k}|<b $ -- это $ d_{k}=0 $, что невозможно. Из противоречия делаем вывод, что представление $ N $ уникально.

 Существенной особенностью всего этого является то, что целое число $ N $ полностью определяется упорядоченным массивом коэффициентов $ a_{m},a_{m-1},\ldots,a_{1},a_{0} $, причем степени $ b $ и знаки плюс являются излишними. Таким образом, число \[ N=a_{m}b^{m}+a_{m-1}b^{m-1}+\ldots+a_{2}b^{2}+a_{1}b+a_{0} \] может быть заменено более простым обозначением \[ N=(a_{m}a_{m-1}\ldots a_{2}a_{1}a_{0})_{b} \] (правая часть не должна интерпретироваться как произведение, она является сокращением для $ N $). Это называется {\it основанием позиционной системы счисления $ b $ для $ N $}.
 
 Малые значения b приводят к длинному представлению чисел, но имеют преимущество в том, что требуют меньшего выбора коэффициентов. Самый простой случай возникает, когда основание $ b=2 $, полученная система счисления называется {\it бинарной системой счисления} (от латинского {\it binarius}, два). Факт в том, что когда число записывается в бинарную систему, то только целые числа 0 и 1 могут появляться в ней как коэффициенты, это означает: каждое положительное целое число можно выразить только одним способом в виде суммы различных степеней 2. Например, целое число 105 может быть записано как \[ 105 = 1\cdot2^{6}+1\cdot2^{5}+0\cdot2^{4}+1\cdot2^{3}+0\cdot2^{2}+0\cdot2+1=2^{6}+2^{5}+2^{3}+1 \] или в сокращенном виде, \[105=(1101001)_{2}.\] В обратной последовательности $(1001111)_{2}$ переводится как \[ 1\cdot2^{6}+0\cdot2^{5}+0\cdot2^{4}+1\cdot2^{3}+1\cdot2^{2}+1\cdot2+1=79.\] Бинарная система наиболее удобна для использования в современных электронных вычислительных машинах, поскольку двоичные числа представлены строками из нулей и единиц; 0 и 1 могут быть выражены в машине переключателем (или аналогичным электронным устройством), который либо выключен, либо включен.
 
 Мы обычно записываем числа в {\it десятичной системе счисления}, где $ b=10 $, опуская нижний индекс 10, который определяет основание. Например, обозначение 1492 представляется более затруднительным выражением \[1\cdot10^{3}+4\cdot10^{2}+9\cdot10+2. \] Целые числа  1, 4, 9 и 2 называются {\it цифрами} заданного числа, 1 -- цифра тысячи, 4 -- цифра сотен, 9 -- цифра десятков и 2 -- цифра единиц. На техническом языке мы называем представление положительных целых чисел суммами степеней 10 с коэффициентами не более 9, как их {\it десятичное представление} (от латинского {\it decem}, десять).
 
 Мы готовы вывести критерии для определения того, делится ли целое число на 9 или 11, не выполняя при этом деления. Для этого нам нужен результат, связанный с конгруэнциями, включающим многочлены с целыми коэффициентами.

\begin{theorem}
	\label{th4-4}
Пусть $P(x)=\sum_{k=0}^mc_{k}x^{k} $ -- полиномиальная функция от  x  с целыми коэффициентами $ c_{k} $. Если  $a \equiv b \pmod{n}$, то $P(a) \equiv P(b) \pmod{n}$.
\end{theorem}

\begin{proof}
Так как $ a \equiv b \pmod{n} $, часть (6) Теоремы 4-2 может быть применена, чтобы получить $a^{k}=b^{k}\pmod{n}$ при $k=0,1\ldots m$. Следовательно, \[ c_{k}a^{k}\equiv c_{k}b^{k}\pmod{n} \] для всех таких k. Добавляя эти m+1 конгруэнции, мы делаем вывод, что \[  \sum \limits_{k=0}^{m}c_{k}a^{k} \equiv \sum\limits_{k=0}^{m}c_{k}b^{k}\pmod{n} \] или, в другом обозначении $P(a) \equiv P(b) \pmod{n}$.
\end{proof}

Если $P(x)$ является полиномом с целыми коэффициентами, то говорят, что a является решением конгруэнтности $ P(x)\equiv 0 \pmod{n} $, если $ P(a)\equiv 0 \pmod{n} $.

\begin{corollary}
Если a является решением $ P(x)\equiv 0 \pmod{n} $ и $a \equiv b \pmod{n}$, то b тоже является решением.	
\end{corollary}

\begin{proof}
Из последней теоремы известно, что $P(a) \equiv P(b) \pmod{n}$. Следовательно, a является решением $ P(x)\equiv 0 \pmod{n} $, тогда $ P(b)\equiv P(a) \equiv 0 \pmod{n}$, что делает b тоже решением.
\end{proof}

Один из тестов на делимость, который мы имеем в виду, заключается в следующем: положительное целое число делится на 9 тогда и только тогда, когда сумма цифр в его десятичном представлении делится на 9.

\begin{theorem}
	\label{th4-5}
Пусть $ N=a_{m}10^{m}+a_{m-1}10^{m-1}+\ldots+a_{1}10+a_{0} $ будет являться десятичным разложением натурального числа N, $ 0\leq a_{k}\leq 10 $ и пусть $ S=a_{0}+a_{1}+\ldots+a_{m} $. Тогда $ 9\mid N $ тогда и только тогда, когда $ 9\mid S $.
\end{theorem}

\begin{proof}
Рассмотрим $ P(x)=\sum_{k=0}^ma_{k}x^{k} $, полином с целыми коэффициентами. Важно отметить, что $ 10\equiv1 \pmod{9} $, откуда по Теореме~\ref{th4-4} $ P(10)\equiv P(1) \pmod{9} $. Но $ P(10)=N $ и $ P(1)=a_{0}+a_{1}+\ldots+a_{m}=S $, поэтому $ N\equiv S \pmod{9} $. Отсюда следует, что  $ N\equiv 0 \pmod{9} $ тогда и только тогда, когда  $ S\equiv 0 \pmod{9} $, что и требовалось доказать.
\end{proof}

Теорема~\ref{th4-4} также служит основой для известного теста делимости на 11; то есть целое число делится на 11 тогда и только тогда, когда знакочередующаяся сумма его цифр делится на 11. Более точно:

\begin{theorem}
	\label{th4-6}
Пусть $ N=a_{m}10^{m}+a_{m-1}10^{m-1}+\ldots+a_{1}10+a_{0} $ будет являться десятичным представлением натурального числа N, $ 0\leq a_{k}\leq 10 $ и пусть $ T=a_{0}-a_{1}+a_{2}-\ldots+(-1)^{m}a_{m} $. Тогда $ 11\mid N $ тогда и только тогда, когда $ 11\mid T $.
\end{theorem}

\begin{proof}
Как и в доказательстве Теоремы~\ref{th4-5}, подставим $ P(x)=\sum_{k=0}^ma_{k}x^{k} $. Так как  $ 10\equiv -1 \pmod{11} $, мы получаем $ P(10)\equiv P(-1) \pmod{11} $. Но  $ P(10)\equiv N $, тогда как  $ P(-1)=a_{0}-a_{1}+a_{2}-\ldots+(-1)^{m}a_{m}=T $, поэтому $ N\equiv T \pmod{11} $. Следовательно, N и T либо делятся на 11, либо нет.
\end{proof}

 \begin{flushleft}
\bf Пример 4-5
\end{flushleft}
Чтобы проиллюстрировать последние два результата, возьмем целое число $ N=1,571,724 $. Поскольку сумма $ 1+5+7+1+7+2+4=27 $ делится на 9, то, исходя из Теоремы~\ref{th4-5}, 9 делит N. Число также может быть разделено на 11; ввиду того, что знакочередующаяся сумма $ 4-2+7-1+7-5+1=11 $ делится на 11.

\begin{center}
	\LARGE {\textsf {\textbf {ПРОБЛЕМЫ 4.3}}}\\[5mm]
\end{center}

\begin{enumerate} 
\item Докажите следующие утверждения:
\begin{enumerate} 
\item Для любого целого числа a цифрой разряда единиц $ a^{2} $ является $ 0,1,2,3,4,5,6,7,8 $ или $ 9 $.
\item Любое из целых чисел $ 0,1,2,3,4,5,6,7,8,9 $ может являться цифрой разряда единиц $ a^{3} $.
\item Для любого целого числа a, цифрой разряда единиц $ a^{4} $ является $ 0,1,5 $ или $ 6 $.
\item Цифрой разряда единиц треугольного числа является $ 0,1,3,5,6 $ или $ 8 $.
\end{enumerate}
\item Найдите последние две цифры числа $ 9^{9^{9}} $. [Подсказка: $ 9^{9^{9}}\equiv 9(\bmod\; 10) $, следовательно $ 9^{9^{9}}=9^{9+10k} $; теперь используйте тот факт, что $ 9^{10}\equiv 1(\bmod\; 100). $ ]
\item Без выполнения операции деления определить, делятся ли целые числа $ 176,521,221 $ и $ 149,235,678 $ на $ 9 $ и $ 11 $.
\item 
\begin{enumerate}
	\item Получить следующее обобщение Теоремы~\ref{th4-5}: если целое число $N$ представлено в основании b как \[N=a_{m}b^{m}+\ldots+a_{2}b^{2}+a_{1}b+a_{0},\ 0\leq a_{k}\leq b-1  \] тогда $ b-1\mid N $ тогда и только тогда, когда $ b-1\mid (a_{m}+\ldots+a_{2}+a_{1}+a_{0}) $.
	\item Установить критерии делимости $ N $ на $3$ и $8$, зависящие от цифр $N$ при написании в основании 9.
	\item Делится ли целое число $ (447836)_{9}$ на $3$ и $8$?
\end{enumerate}
\item Используя тест 9 или тест 11, найдите пропущенные цифры в расчетах ниже:
\begin{enumerate}
\item $ 52817\cdot3212146=169655x15282 $;
\item $2x99561=[3(523+x)]^{2}$.	
\end{enumerate}
\item Установить следующие критерии делимости:
\begin{enumerate}
\item Целое число делится на 2 тогда и только тогда, когда его цифра разряда единиц равна $ 0,2,4,6 $ или $8$.
\item Целое число делится на 3 тогда и только тогда, когда сумма его цифр делится на 3.
\item Целое число делится на 4 тогда и только тогда, когда число, образованное от его цифры разряда десятков и единиц, делится на 4. [Подсказка: $10^{k}\equiv0(\bmod\; 4)$ для $ k\geq2 $.]
\end{enumerate}
\item Показать, что $ 2^{n} $ делит число $N$ тогда и только тогда, когда  $ 2^{n} $ делит число, состоящее из последних $n$ цифр из $N$. [Подсказка: $10^{k}=2^{k}5^{k}\equiv0(\bmod\; 2^{n}) $ для $ k\geq n $.]
\item Без выполнения операции деления определить, делится ли целое число $1,010,908,899 $ на $ 7,11 $ и $13$.
\item 
\begin{enumerate}
\item Дано целое число $N$, пусть $M$ будет целым числом, образованным путем изменения порядка цифр $N$ (Например, если $N=6923$, тогда $M=3296$).\\ Проверьте, делится ли $N-M$ на $9$.
\end{enumerate}
\end{enumerate}
\end{document}